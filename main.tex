\documentclass[twoside]{svproc}

\usepackage{polyglossia}
\setmainlanguage[variant=american]{english}
\usepackage{fontspec}
\usepackage[onehalfspacing]{setspace}

\usepackage{microtype}
\usepackage{changepage}

\makeatletter
\newcommand{\@chapapp}{\relax}%
\makeatother
\usepackage[page]{appendix}
\renewcommand{\appendixpagename}{Appendix:\\Evaluation Results}

\usepackage{blindtext, lipsum, graphicx}
\graphicspath{{./figs/}}

\setlength\parindent{0pt}
\frenchspacing

\usepackage{algorithm2e}

\usepackage{tabu}

\usepackage[cmex10]{amsmath}
% define custom math operators if you like
% \DeclareMathOperator*{\argmax}{argmax}

\begin{document}

\setlength\parskip{7.5pt}
%\setlength\intextsep{0pt}

\title{Project Title}
\titlerunning{The running project title}  % abbreviated title (for running head)

\author{
	Lastname1, Firstname1 \\
	\textit{lastname1@uni-potsdam.de} \\
	Lastname2, Firstname2 \\
	\textit{lastname1@uni-potsdam.de}
}
\institute{Master's Program in \emph{Cognitive Systems}\\
University of Potsdam}

% use for special paper notices
%\IEEEspecialpapernotice{(Individual Research Project)}

% make the title area
\maketitle


\begin{abstract}
\blindtext[2]
\end{abstract}

% Note that keywords are not normally used for peerreview papers.
%\begin{keywords}
%deep learning, reinforcement learning, language understanding
%\end{keywords}

\clearpage
\section{Introduction}
\blindtext[4]
\section{Basics}
\label{sec:basics}
\Blindtext

\section{Project}
\label{sec:project}
\Blindtext
\section{Evaluation}
\label{sec:evaluation}
\blindtext[2]

\begin{table}[t]
    {\sffamily
    \begin{tabu} to \linewidth { X[c] | X[c] X[c] X[c] }
        \tabucline[1pt]{-}
        \textbf{outcome} & \textbf{sleepy} & \textbf{bored} & \textbf{hungry} \\ \hline 
        \textbf{success} & 12.98~\%   & 0.0~\%     & 10.84~\%   \\
        \textbf{timeout} & 10.91~\%   & 2.18~\%   & 12.38~\%   \\
        \textbf{failure} & 9.63~\%    & 30.90~\%  & 10.17~\%   \\ \tabucline[1pt]{-}
    \end{tabu}}
    \vspace*{7pt}
    \caption{\blindtext}
    \label{tab:train_full_env}
\end{table}

\section{Conclusions}
\label{sec:conclusions}

\blindtext[2]


\bibliographystyle{plain}
\bibliography{paperrefs}

\begin{appendices}
\blindtext
\end{appendices}

\end{document}